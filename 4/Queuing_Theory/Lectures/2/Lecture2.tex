\documentclass{article}

\usepackage{indentfirst}
\setlength{\parindent}{5ex}

\usepackage[utf8]{inputenc}
\usepackage[english,russian]{babel}

\usepackage{amsmath}
\usepackage{amsfonts}
\usepackage{amssymb}
\usepackage{amsthm}

\usepackage[left=2cm,right=2cm,top=2cm,bottom=2cm]{geometry}

\usepackage[pdftex]{graphicx}

\usepackage[dvipsnames]{color}
\usepackage{colortbl}

\usepackage{wrapfig}
\usepackage{caption}
\usepackage{subcaption}
\graphicspath{{pictures/}}
\DeclareGraphicsExtensions{.pdf,.png,.jpg}
\usepackage{float}

\usepackage{diagbox}

\newtheorem*{theorem*}{Theorem}
\newtheorem*{lemma*}{Lemma}
\newtheorem*{remark*}{Remark}

\title{\textbf{Лекция 2\\по Теории Массового Обслуживания}}
\author{\textit{от Татаринова Никиты Алексеевича, БПИ193}}
\date{2022.09.15}

\newcommand\twodigits[1]{%
   \ifnum#1<10 0#1\else #1\fi
}

\newcommand*\rfrac[2]{{}^{#1}\!/_{#2}}

\begin{document}
\maketitle
\section*{Введение в дифференциальные и разностные уравнения}
\subsection*{Примеры разностных уравнений}
\begin{equation*}
\left\{
\begin{aligned}
& P(t) \quad \text{-- численность омёб в момент } t \\
& P(t) \, = \, 2 \! \cdot \! P(t) \\
& P(0) \! = \! 100
\end{aligned}
\right| \quad \text{разностное уравнение 1го порядка} 
\end{equation*}
\[ y(t) \, = \, 2 \! - \! y(t \! - \! 1) \! \! + \! 0.5 \! \cdot \! y(t \! - \! 2) \quad \text{разностное уравнение 2го порядка} \]\par
Это было в школе (1):
\[ y(k) \, = \, y(k \! - \! 1) \! + \! b \quad \rightarrow \quad y(k) \, = \, y(0) \! + \! k \! \cdot \! b \]
\[ y(k) \, = \, C \! + \! k \! \cdot \! b \quad \text{-- общее решение разностного уравнения} \]
\[ y(k) \, = \, 3 \! + \! k \! \cdot \! b \quad \text{-- частное решение, удовлетворяющее условию } y(0) \! = \! 3 \]\par
Это было в школе (2):
\[ y(k) \, = \, a \! \cdot \! y(k \! - \! 1) \quad \rightarrow \quad y(k) \, = \, a^k \! \cdot \! y(0) \]
\[ y(k) \, = \, C \! \cdot \! a^k \quad \text{-- общее решение разностного уравнения} \]
Этого в школе не было, но сводится:
\[ y(k) \, = \, a \! \cdot \! y(k \! - \! 1) \! + \! b \quad \rightarrow \quad y(k) \, = \, a^k \! \cdot \! y(0) + b \! \cdot \! \dfrac{a^k \! - \! 1}{a \! - \! 1} \]
\[ y(k) \, = \, C \! \cdot \! a^k \! + \! b \! \cdot \! \dfrac{a^k \! - \! 1}{a \! - \! 1} \quad \text{-- общее решение разностного уравнения} \]
При $ |a| \! < \! 0 $:
\[ a^k \, \underset{k \rightarrow +\infty}{\longrightarrow} 0 \]
\[ y(k) \, \longrightarrow \, \dfrac{b}{1 \! - \! a}  \]
При $ |a| \! > \! 0 $:
\[ \nexists \lim\limits_{k \rightarrow +\infty} y(k) \]
При $ |a| \! = \! 1 $?\par
Стационарное решение:
\[ y(k) \, = \, y(k \! - \! 1) \, = \, y(k \! - \! 2) \, = \, ... \]
\subsection*{Примеры дифференциальных уравнений}
\[ \dfrac{dy}{dx} \, = \, y \! + \! 3 \]
\[ y \! \cdot \! \dfrac{dy}{dx} \, = \, 50 \! \cdot \! x^3 \]
\[ x \! \cdot \! \dfrac{dy}{dx} \! - \! 2y \, = \, 2 \! \cdot \! x^4 \]\par
Пример решения:
\[ \dfrac{dy}{dx} \, = \, 2 \! \cdot \! x \]
\[ \int \dfrac{dy}{dx} \! \cdot \! dx \, = \, \int 2 \! \cdot \! x \! \cdot \! dx \! + \! C \]
\[ y \, = \, x^2 \! + \! C \quad \text{-- общее решение дифференциального уравнения} \]
Начальное условие $ y(0) \! = \! 12 $:
\[ y(0) \, = \, C \quad \rightarrow \quad C \! = \! 12 \]
\[ y \! = \! x^2 \! + \! 12 \quad \text{-- частное решение дифференциального уравнения} \]\par
Школьный пример:
\begin{equation*}
\left\{
\begin{aligned}
& y(t) \quad \text{-- высота шарика в момент } t \\
& y(0) \, = \, 50 \\
& \overset{\cdot}{y}(0) \, = \, 0 \\
& \overset{\cdot \cdot}{y} \, = \, -9.81
\end{aligned}
\right.
\end{equation*}
Ищем скорость:
\[ \overset{\cdot}{y} \, = \, \int \overset{\cdot \cdot}{y} \! \cdot \! dt \, + \, C \, = \, \int (-9.81) \! \cdot \! dt \, + \, C \, = \, (-9.81) \! \cdot \! t \, + \, C \]
Из начальных условий:
\[ 0 \, = \, \overset{\cdot}{y}(0) \, = \, (-9.81) \! \cdot \! 0 \! + \! C \, = \, C \]
Ищем высоту:
\[ y \, = \, \int \overset{\cdot}{y} \! \cdot \! dt \, + \, C \, = \, \int (-9.81) \! \cdot \! t \! \cdot \! dt \, + \, C \, = \, \left( -\dfrac{9.81}{2} \right) \! \cdot \! t^2 \, + \, C \]
Из начальных условий:
\[ 50 \, = \, y(0) \, = \left( -\dfrac{9.81}{2} \right) \! \cdot \! 0^2 \, + \, C \, = \, C \]
Частное решение:
\[ y(t) \, = \, \left( -\dfrac{9.81}{2} \right) \! \cdot \! t^2 \! + \! 50 \]
\subsection*{Дифференциальные уравнения с разделяющимися переменными}
\[ \dfrac{dy}{dx} \! \cdot \! \varphi (y) \, = \, f(x) \]\par
Схема решения:
\[ \varphi(y) \! \cdot \! dy \, = \, f(x) \! \cdot \! dx \]
\[ \int \varphi (y) \! \cdot \! dy \, = \, \int f(x) \! \cdot \! dx \, + \, C \]
\subsection*{Линейное дифференциальное уравнение (метод вариации произвольной постоянной)}
\[ \dfrac{dy}{dx} \! + \! a(x) \! \cdot \! y \, = \, f(x) \]
Интегрирующий множитель:
\[ u(x) \, = \, e^{\int a(x) \cdot dx} \]
Домножаем:
\[ \dfrac{dy}{dx} \! \cdot \! u(x) \! + \! a(x) \! \cdot \! u(x) \! \cdot \! y \, = \, f(x) \! \cdot \! u(x) \]
Пусть $ g(x) \, = \, y(x) \! \cdot \! u(x) $. Тогда:
\[ \dfrac{dg}{dx} \, = \, f(x) \! \cdot \! u(x) \]
\[ g(x) \, = \, \int dg \, = \, \int f(x) \! \cdot \! u(x) \! \cdot \! dx \, + \, C \]
\[ y \, = \, \dfrac{g(x)}{u(x)} \]
\end{document}