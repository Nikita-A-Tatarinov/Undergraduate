\documentclass{article}

\usepackage{indentfirst}
\setlength{\parindent}{5ex}

\usepackage[utf8]{inputenc}
\usepackage[english,russian]{babel}

\usepackage{amsmath}
\usepackage{amsfonts}
\usepackage{amssymb}
\usepackage{amsthm}

\usepackage[left=2cm,right=2cm,top=2cm,bottom=2cm]{geometry}

\usepackage[pdftex]{graphicx}

\usepackage[dvipsnames]{color}
\usepackage{colortbl}

\usepackage{wrapfig}
\usepackage{caption}
\usepackage{subcaption}
\graphicspath{{pictures/}}
\DeclareGraphicsExtensions{.pdf,.png,.jpg}
\usepackage{float}

\usepackage{diagbox}

\newtheorem*{theorem*}{Theorem}
\newtheorem*{lemma*}{Lemma}
\newtheorem*{remark*}{Remark}

\title{\textbf{Лекция 1\\по Теории Массового Обслуживания}}
\author{\textit{от Татаринова Никиты Алексеевича, БПИ193}}
\date{\the\year .\twodigits{\the\month}.\twodigits{\the\day}}
\newcommand\twodigits[1]{%
   \ifnum#1<10 0#1\else #1\fi
}

\newcommand*\rfrac[2]{{}^{#1}\!/_{#2}}

\begin{document}
\maketitle
\section*{Литература}
\begin{itemize}
\item D.Cross, C.M.Harris "Fundamentals of Queuing Theory"
\end{itemize}
\section*{Теория массового обслуживания}
Родилась из нужд телекоммуникационных компаний -- организовать телекоммуникации оптимальным образом. Совмещает теорию вероятностей и исследование операций. 1909г -- датский инженер А.К.Эрланг описал модель появления событий во времени -- "Theory of Probability and Telephone Conversations".\par
Что интересует?
\begin{itemize}
\item частота звонков (или среднее время между звонками) -- интенсивность входящего потока звонков
\item среднее время на обслуживание звонка -- интенсивность обслуживания
\item число телефонисток
\end{itemize}\par
\subsection*{Модель}
\[
\begin{array}{ccccccc}
 & & & \nearrow & \framebox{канал 1} & \searrow & \\
 
\text{входящий поток заявок} & \rightarrow & \framebox{очередь} & \rightarrow & \vdots & \rightarrow & \text{обслуженные заявки} \\

 & & \downarrow & \searrow & \framebox{канал $ n $} & \nearrow & \\
 & & \text{отказ}
\end{array} 
\]
\subsection*{Нотация Кендалла}
Последовательность из нескольких символов и цифр:
\[ A/B/X/Y/Z \]
, который характеризуют некоторые особенности системы массового обслуживания.
\begin{itemize}
\item $ A $ -- закон распределения времени между поступлением заявок
\begin{itemize}
\item $ M $ -- экспоненциальное (беспамятное распределение)
\item $ E_k / Er_k $ -- распределение Эрланга порядка $ k $
\item $ D $ -- детерминированное время
\item $ G $ -- произвольное распределение (отсутствие ограничений)
\end{itemize}
\item $ B $ -- закон распределения времени обслуживания (те же обозначения)
\item $ X $ -- число каналов обслуживания (может быть бесконечным, но нетипично)
\item $ Y $ -- ёмкость системы -- максимальное число заявок в системе единовременно (может быть бесконечным, но нетипично)
\item $ Z $ -- дисциплина обслуживания -- показывает, в каком порядке обслуживаются заявки
\begin{itemize}
\item $ FCFS / FIFO $ -- First Come First Served
\item $ LCFS / LIFO $ -- Last Come First Served
\item $ RSS / SIRO $ -- Random Selection for Service
\item $ PS $ -- processor sharing (например, 2 канала обслуживают одновременно, но в 2 раза медленнее)
\item $ PRI $ -- приоритетное обслуживание (заявки разбиваются на разные типы с разными приоритетами)
\item $ GD $ -- general discipline
\end{itemize}
\end{itemize}
\subsection*{Основные случайные величины, связанные с процессом обслуживания}
Для числа заявок:
\begin{equation*}
\left\{
\begin{aligned}
& \underset{\substack{\text{число заявок} \\ \text{в системе}}}{N} \! = \! \underset{\substack{\text{число заявок} \\ \text{в очереди}}}{N_q} \! + \! \underset{\substack{\text{число заявок} \\ \text{на обслуживании}}}{N_s} \\
& L \! = \! E \big[ N \big] \quad \text{-- среднее число заявок в системе} \\
& L_q \! = \! E \big[ N_q \big] \quad \text{-- среднее число заявок в очереди} \\
& L_s \! = \! U \! = \! E \big[ N_s \big] \quad \text{-- коэффициент загрузки мощностей} \\
\end{aligned}
\right.
\end{equation*}\par
Для времени:
\begin{equation*}
\left\{
\begin{aligned}
& \underset{\substack{\text{время пребывания} \\ \text{заявки в системе}}}{T} \! = \! \underset{\substack{\text{время ожидания} \\ \text{в очереди}}}{T_q} \! + \! \underset{\substack{\text{время} \\ \text{обслуживания}}}{T_s} \\
& W \! = \! E \big[ T \big] \quad \text{-- среднее время в системе} \\
& W_q \! = \! E \big[ T_q \big] \quad \text{-- среднее время ожидания в очереди} \\
& \mu \! = \! \dfrac{1}{E \big[ N_s \big]} \quad \text{-- интенсивность обслуживания} \\
\end{aligned}
\right.
\end{equation*}\par
Кроме того:
\begin{equation*}
\left\{
\begin{aligned}
& \mathbf{P}_{loss} \quad \text{-- вероятность потери} \\
& \lambda \quad \text{-- интенсивность входящего потока заявок} \\
& ? \quad \text{-- пропускная способность}
\end{aligned}
\right.
\end{equation*}\par
\subsection*{Пример}
Заявки поступают в моменты $ 0,20,40,... $. Время обслуживания равномерное на отрезке $ [14;24] $ минут. Один канал обслуживания. Если заявка поступает, а канал занят, то она получает отказ. Найти вероятность $ P_l(k) $ потери заявки с номером $ k $.
\[ \mathbf{P}_l(1) \! = \! 0 \]
\[ \mathbf{P}_l(2) \! = \! \dfrac{24 \! - \! 20}{24 \! - \! 14} \! = \! 0.4 \]
\[ \mathbf{P}_l(3) \! = \! \big( 1 \! - \! P_l(2) \big) \! \cdot \! 0.4 \! = \! 0.24 \]
\[ \mathbf{P}_l(4) \! = \! \big( 1 \! - \! P_l(3) \big) \! \cdot \! 0.4 \! = \! 0.304 \]
\[ \mathbf{P}_l(k) \! = \! \big( 1 \! - \! P_l(k \! - \! 1) \big) \! \cdot \! 0.4 \! \]\par
График представляет затухающие колебания вокруг некоего значения. Первые этапы называются переходным периодом, дальнейшие - стационарным периодом.
\section*{Геометрическая интерпретация математического ожидания}
\begin{equation*}
\begin{aligned}
& G_T (t) \, = \, \mathbf{P} \big\{ T \! > \! t \big\} \, = \, 1 \! - \! F_T(t) \qquad & \text{-- дополнительная функция распределения} \\
& & \text{(complementary cdf)}
\end{aligned}
\end{equation*}\par
Тогда:
\[ E \big[ T \big] \, = \, \int\limits_0^{+\infty} G_T(t) \! \cdot \! dt \, - \, \int\limits_{-\infty}^0 F_T(t) \! \cdot \! dt \]
Для неотрицательной СВ $ T $:
\[ E \big[ T \big] \, = \, \int\limits_0^{+\infty} G_T(t) \! \cdot \! dt \]
\subsection*{Пример 1}
Пусть $ T \! \sim \! Exp(\lambda) $, т.е.
\begin{equation*}
G_T(t) \, = \, 1 \! - \! F_T(t) \, = \,
\left\{
\begin{aligned}
& 1 & & \quad t \! < \! 0 \\
& e^{-\lambda t} & & \quad t \! \geqslant \! 0
\end{aligned}
\right.
\end{equation*}\par
Тогда:
\[ E \big[ T \big] \, = \, \int\limits_0^{+\infty} e^{-\lambda t} \! \cdot \! dt \, = \, \left. -\dfrac{1}{\lambda} \! \cdot \! e^{-\lambda t} \right|_0^{+\infty} \, = \, 0 \! - \! \left( -\dfrac{1}{\lambda} \right) \, = \, \dfrac{1}{\lambda} \]
\subsection*{Пример 2}
Пусть:
\[ G_T(t) \, = \, 0.7 \! \cdot \! e^{-t} \qquad t \! \geqslant \! 0 \]\par
Найти:
\begin{enumerate}
\item[а)] долю заявок, которым приходится ждать в очереди
\item[б)] среднее время ожидания
\item[в)] вероятность, что заявке придётся ждать более 2 минут при условии, что ей придётся ждать
\end{enumerate}
\subsubsection*{Решение}
\begin{enumerate}
\item[а)]
\[ \mathbf{P} \big\{ T \! > \! 0 \big\} \, = \, G_T(0) \, = \, 0.7 \]
\item[б)]
\[ E \big[ T \big] \, = \, \int\limits_0^{+\infty} 0.7 \! \cdot \! e^{-t} \! \cdot \! dt \, = \, -0.7 \! \cdot \! e^{-t} \Big|_0^{+\infty} \, = \, 0 \! - \! (-0.7) \, = \, 0.7 \]
\item[в)]
\[ \mathbf{P} \bigg( \big\{ T \! > \! 2 \, \big\} \Big| \big\{ \, T \! > \! 0 \big\} \bigg) \, = \, \dfrac{\mathbf{P} \bigg( \big\{ T \! > \! 2 \, \big\} \bigcap \big\{ \, T \! > \! 0 \big\} \bigg)}{\mathbf{P} \big\{ T \! > \! 0 \big\}} \, = \, \dfrac{\mathbf{P} \big\{ T \! > \! 2 \big\}}{\mathbf{P} \big\{ T \! > \! 0 \big\}} \, = \, \dfrac{G_T(2)}{G_T(0)} \, = \, e^{-2} \]
\end{enumerate}
\end{document}