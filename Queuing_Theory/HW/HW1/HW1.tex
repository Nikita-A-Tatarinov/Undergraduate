\documentclass{article}

\usepackage{indentfirst}
\setlength{\parindent}{5ex}

\usepackage[utf8]{inputenc}
\usepackage[english,russian]{babel}

\usepackage{amsmath}
\usepackage{amsfonts}
\usepackage{amssymb}
\usepackage{amsthm}

\usepackage[left=2cm,right=2cm,top=2cm,bottom=2cm]{geometry}

\usepackage[pdftex]{graphicx}

\usepackage[dvipsnames]{color}
\usepackage{colortbl}

\usepackage{wrapfig}
\usepackage{caption}
\usepackage{subcaption}
\graphicspath{{pictures/}}
\DeclareGraphicsExtensions{.pdf,.png,.jpg}
\usepackage{float}

\usepackage{diagbox}

\newtheorem*{theorem*}{Theorem}
\newtheorem*{lemma*}{Lemma}
\newtheorem*{remark*}{Remark}

\title{\textbf{ДЗ 1\\по Теории Массового Обслуживания}}
\author{\textit{от Татаринова Никиты Алексеевича, БПИ193}}
\date{\the\year .\twodigits{\the\month}.\twodigits{\the\day}}
\newcommand\twodigits[1]{%
   \ifnum#1<10 0#1\else #1\fi
}

\newcommand*\rfrac[2]{{}^{#1}\!/_{#2}}

\begin{document}
\maketitle
\section*{Задание 1}
\subsection*{Условие}
В столовой посетители обслуживаются на двух кассах, время обслуживания каждого посетителя распределено по закону Эрланга второго порядка и составляет в среднем одну минуту. Покупатели подходят с интенсивностью два человека в минуту, время между покупателями распределено экспоненциально. Очередь к кассам может быть сколь угодно большой.
Опишите систему нотацией Кендалла.
\subsection*{Решение}
Нотация Кендалла имеет вид:
\[ A/B/X/Y/Z \]
, где 
\begin{itemize}
\item $ A $ -- закон распределения времени между поступлением заявок, то есть $ A \! = \! M $ (экспоненциальное);
\item $ B $ -- закон распределения времени обслуживания, то есть $ B \! = \! E_2 $ (Эрланга второго порядка);
\item $ X $ -- число каналов обслуживания, то есть $ X \! = \! 2 $ (две кассы);
\item $ Y $ -- ёмкость системы (максимальное число заявок в системе единовременно), то есть $ Y \! = \! \infty $ (очередь к кассам может быть сколь угодно большой);
\item $ Z $ -- дисциплина обслуживания, то есть $ Z \! = \! GD $ (произвольная дисциплина, так как иное не оговорено условием).
\end{itemize}
\subsection*{Ответ}
\[ M/E_2/2/\infty/GD \quad \sim \quad M/E_2/2 \]
\section*{Задание 2}
\subsection*{Условие}
Опираясь на геометрическую интерпретацию, найдите математическое ожидание случайной величины X, если
\begin{enumerate}
\item[а)] эта величина имеет геометрическое распределение с параметром $ p \! \in \! (0;1) $, то есть принимает значения $ 0,1,2... $ с вероятностями $ \mathbf{P} \big\{ X \! = \! x \big\} \, = \, {(1 \! - \! p)}^x \! \cdot \! p $
\item[б)] эта величина имеет функцию распределения
\begin{equation*}
F(x) \, = \,
\left\{
\begin{aligned}
& 0 & & \quad x \! < \! 0 \\
& 1 \! - \! 0.8 \! \cdot \! e^{-x} & & \quad x \! \geqslant \! 0
\end{aligned}
\right.
\end{equation*}
\end{enumerate}
\subsection*{Решение}
Дополнительная функция распределения:
\[ G_X(x) \, = \, \mathbf{P} \big\{ X \! > \! x \big\} \, = \, 1 \! - \! F_X(x) \]
Тогда для неотрицательных $ X $:
\[ E \big[ X \big] \, = \, \int\limits_0^{+\infty} G_X(x) \! \cdot \! dx \]
\begin{enumerate}
\item[а)]
\[ E \big[ X \big] \, = \, \sum\limits_{i=1}^{+\infty} i \! \cdot \! p \! \cdot \! {(1 \! - \! p)}^i \, = \, \sum\limits_{i=1}^{+\infty} (i \! + \! 1) \! \cdot \! p \! \cdot \! {(1 \! - \! p)}^i \, - \, \sum\limits_{i=1}^{+\infty} p \! \cdot \! {(1 \! - \! p)}^i \, = \, p \sum\limits_{i=1}^{+\infty} \dfrac{dq^{i+1}}{dq} \, - \, (1 \! - \! p) \, = \]
\[ = \, p \! \cdot \! \dfrac{d}{dq} \left( \sum\limits_{i=1}^{+\infty} q^{i+1} \right) \, - \, (1 \! - \! p) \, = \, p \! \cdot \! \dfrac{d}{dq} \left( \dfrac{q^2}{1 \! - \! q} \right) \, - \, (1 \! - \! p) \, = \, p \! \cdot \! \dfrac{2 \! \cdot \! q \! \cdot \! (1 \! - \! q) \! - \! q^2 \! \cdot \! (-1)}{{(1 \! - \! q)}^2} \, - \, (1 \! - \! p) \, = \]
\[ = \, p \! \cdot \! \dfrac{(2 \! - \! q) \! \cdot \! q}{{(1 \! - \! q)}^2} \, - \, (1 \! - \! p) \, = \, p \! \cdot \! \dfrac{(1 \! + \! p) (1 \! - \! p)}{p^2} \! - \! (1 \! - \! p) \, = \, \dfrac{1 \! - \! p^2}{p} \! + \! p \! - \! 1 \, = \, \dfrac{1}{p} \! - \! 1 \]
\item[б)]
\[ E \big[ X \big] \, = \, \int\limits_0^{+\infty} 0.8 \! \cdot \! e^{-x} \! \cdot \! dx \, = \, -0.8 \! \cdot \! e^{-x} \Big|_0^{+\infty} \, = \, 0 \! - \! (-0.8) \, = \, 0.8 \]
\end{enumerate}
\subsection*{Ответ}
\begin{enumerate}
\item[а)] $ E \big[ X \big] \, = \, \dfrac{1}{p} \! - \! 1 $
\item[б)] $ E \big[ X \big] \, = \, 0.8 $
\end{enumerate}
\section*{Задание 3}
\subsection*{Условие}
Consider a system with a single server at which customers arrive with interval times distributed uniformly from $ 0.7 $ to $ 1.2 $ minutes. Each customer needs exactly $ 1 $ min to be serviced. If a customer finds the server busy upon arrival, it leaves without being serviced (it is lost). Let $ \mathbf{P} (k) $ denote the probability that $ k $-th customer is lost. The server is initially unoccupied, so that $ \mathbf{P} (1) \! = \! 0 $.\par
Calculate $ \mathbf{P} (k) $ for $ k \! = \! 2,3,4 $ and find $ \mathbf{P} (k) $.
\subsection*{Решение}
Из условия, $ 1 $-я заявка точно не будет потеряна. Поскольку каждая заявка обслуживается ровно 1 минуту, $ 2 $-я заявка будет потеряна, если придёт через $ [0.7;1] $ минуту после $ 1 $й, то есть:
\[ \mathbf{P} (2) \! = \! \dfrac{1 \! - \! 0.7}{1.2 \! - \! 0.7} \, = \, 0.6 \]\par
Далее рассмотрим произвольную $ k $-ю заявку $ \forall k \! \geqslant \! 3 $. Если $ (k \! - \! 1) $-я заявка потеряна, то $ k $-я заявка точно не будет потеряна, так как она придёт минимум через $ 0.7 $ минут после $ (k \! - \! 1) $-й, а $ (k \! - \! 1) $-я заявка пришла минимум через $ 0.7 $ минут после $ (k \! - \! 2) $-й, то есть $ k $-я заявка придёт минимум через $ 1.4 $ минуты после $ (k \! - \! 2) $, в то время как обслуживание любой заявки занимает ровно минуту. Если же $ (k \! - \! 1) $-я заявка не потеряна, то вероятность потери $ k $й рассчитывается так же, как и для $ 2 $й. Таким образом, вероятность потери $ k $-й заявки равна
\[ \mathbf{P}(k) \, = \, \Big( 1 \! - \! \mathbf{P}(k \! - \! 1) \Big) \! \cdot \! \mathbf{P}(2) \]
Для $ 3 $-й заявки:
\[ \mathbf{P}(3) \, = \, (1 \! - \! 0.6) \! \cdot \! 0.6 \, = \, 0.24 \]
Для $ 2 $-й заявки:
\[ \mathbf{P}(4) \, = \, (1 \! - \! 0.24) \! \cdot \! 0.6 \, = \, 0.456 \]
\subsection*{Ответ}
\begin{equation*}
\left\{
\begin{aligned}
& \mathbf{P}(2) \! = \! 0.6 \\
& \mathbf{P}(3) \! = \! 0.24 \\
& \mathbf{P}(4) \! = \! 0.456 \\
& \mathbf{P}(k) \! = \! \Big( 1 \! - \! \mathbf{P}(k \! - \! 1) \Big) \! \cdot \! \mathbf{P} (2) \quad \forall k \! \geqslant \! 3 \\
\end{aligned}
\right.
\end{equation*}
\end{document}